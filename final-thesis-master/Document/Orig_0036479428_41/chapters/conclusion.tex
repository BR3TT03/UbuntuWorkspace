Tehnika prepoznavanja tiskanih i rukom pisanih slova se razvija još od polovice prošlog stoljeća i od tada je u mnogočemu evoluirala. U ovom radu, kao klasifikator pojedinog slova, korištena je višeslojna unaprijedna potpuno povezana neuronska mreža učena s algoritmom propagacije pogreške unatrag uz dodatak momenta inercije.

U radu je predstavljen cjeloviti postupak obrade slike kako bi se ulazna slika pripremila za izlučivanje značajki i učenje mreže. Kao vektor značajki korišten je hibridni vektor sastavljen od dijagonalne i horizontalne projekcije te broja presjecišta i broja krajnjih točaka.

Dobiveni rezultati su u granicama očekivanoga za dani skup podataka. U budućnosti bi se mogao proširiti skup podataka za zahtjevniju uporabu (trenutno se sastoji od 4480 slova). Također, kao klasifikator bi se mogla isprobati i duboka konvolucijska neuronska mreža koja problem prepoznavanja znamenaka rješava s točnošću od preko $99.7$\% \citep{umjint}.