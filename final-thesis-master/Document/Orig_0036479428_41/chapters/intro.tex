Zahtjev za svakodnevnom obradom sve veće količine podataka polako, ali sigurno, stvara velik problem čovjeku. Svaki dan zahtjeva se obrada ogromne količine papirnatih dokumenata koje je uglavnom potrebno obraditi i unijeti na računalo. Kako bi se izbjeglo ručno prepisivanje projektiraju se raznorazni sustavi koji će taj posao automatizirati.

Primjer jednog takvog posla je i ispravljanje provjera znanja s abc-pitalicama. Ispitanik svoj odgovor unosi zacrnjivanjem u matrici za unos odgovora. No u slučaju pogreške, ispitanik može promijeniti odgovor tako da izravno upiše novi odgovor na za to odgovarajuće mjesto. Ukoliko bi jedan ispravljač ručno provjeravao svaki obrazac i unosio u sustav, taj bi posao na kolegijima od 600 i više studenata trajao i po nekoliko dana.

Kroz ovaj rad pokazat će se implementacija sustava za prepoznavanje rukom pisanih slova koji će na svoj ulaz primiti izdvojenu sliku slova te kao rezultat vratiti prepoznato slovo. Prvi korak će biti prikupljanje skupa podataka rukom pisanih slova, zatim obrada istih. Nakon toga slijedi izgradnja sustava za klasifikaciju pojedinog slova te njegovo iskorištavanje u stvarnim uvjetima rada.

Kao sustav za klasifikaciju bit će korištena umjetna neuronska mreža učena algoritmom propagacije pogreške unatrag. Razlog tomu je jednostavna implementacija takvog sustava te velika sposobnost generalizacije koje nudi neuronska mreža. Kako dobivena slika slova može biti velika i neobrađena, prije prikazivanja podataka neuronskoj mreži, sliku je potrebno obraditi te izvući vektor značajki. Prilikom izlučivanja vektora značajki koristiti će se dijagonalna projekcija, koja je relativno nov pojam predstavljen u radu \citep{diagonal2011}. 

Ostatak rada organiziran je kako slijedi. U drugom poglavlju opisan je postupak prikupljanja i obrade skupa podataka. U trećem poglavlju opisan je postupak klasifikacije korištenjem neuronske mreže koja je učena algoritmom propagacije pogreške unatrag uz dodatak momenta inercije. U četvrtom poglavlju su predstavljene metode izlučivanja značajki iz pojedinog slova dok su u petom poglavlju predstavljeni usporedni rezultati učenja neuronske mreže korištenjem različitih metoda izlučivanja značajki.